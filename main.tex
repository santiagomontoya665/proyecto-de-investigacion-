\documentclass{article}
\usepackage[utf8]{inputenc}
\usepackage[spanish]{babel}
\usepackage{listings}
\usepackage{graphicx}
\graphicspath{ {images/} }
\usepackage{cite}

\begin{document}

\begin{titlepage}
    \begin{center}
        \vspace*{1cm}
            
        \Huge
        \textbf{Taller Memoria}
            
        \vspace{0.5cm}
        \LARGE
        Subtítulo
            
        \vspace{1.5cm}
            
        \textbf{Santiago Montoya Leal}
            
        \vfill
            
        \vspace{0.8cm}
            
        \Large
        Despartamento de Ingeniería Electrónica y Telecomunicaciones\\
        Universidad de Antioquia\\
        Medellín\\
        Septiembre de 2020
            
    \end{center}
\end{titlepage}

\tableofcontents
\newpage
\section{Introducción}\label{intro}
En este documento trata sobre los tipos de memoria del computador y su funcionalidad con los cuales se busca responder el taller – nociones de la memoria del computador: ¿Qué es la memoria de un computador?, mencione los tipos de memoria que conoce y haga una pequeña descripción de cada tipo, describa la manera de cómo se gestiona la memoria de un computador y ¿Qué hace que la memoria sea más rápida que otra? ¿Porque esto es importante?

\section{Preguntas del taller -nociones de memoria del computador} \label{contenido}
En esta sección se busca responder las 4 preguntas que están en el taller de nociones de memoria del computador
\subsection{Citación}
Vamos a citar por ejemplo un artículo de \textbf{Albert Einstein} \cite{einstein}.
También es posible citar libros \cite{dirac} o documentos en línea \cite{knuthwebsite}.\\\\
Revisar en la última sección el formato de las referencias en IEEE.

\subsection{Incluir código en el documento}
%
A continuación, se presenta el código \ref{codigo_ejemplo}, que nos permite incluir en el informe partes de código que requieran una explicación exhaustiva.
\begin{lstlisting}[language=C++, caption=Ejemplo, label=codigo_ejemplo]
#include <stdio.h>
#define N 10
/* Block
 * comment */

int main()
{
    int i;

    // Line comment.
    puts("Hello world!");
    
    for (i = 0; i < N; i++)
    {
        puts("LaTeX is also great for programmers!");
    }

    return 0;
}
\end{lstlisting}





\bibliographystyle{IEEEtran}
\bibliography{references}

\end{document}
